% Options for packages loaded elsewhere
% Options for packages loaded elsewhere
\PassOptionsToPackage{unicode}{hyperref}
\PassOptionsToPackage{hyphens}{url}
\PassOptionsToPackage{dvipsnames,svgnames,x11names}{xcolor}
%
\documentclass[
  letterpaper,
  DIV=11,
  numbers=noendperiod]{scrreprt}
\usepackage{xcolor}
\usepackage{amsmath,amssymb}
\setcounter{secnumdepth}{5}
\usepackage{iftex}
\ifPDFTeX
  \usepackage[T1]{fontenc}
  \usepackage[utf8]{inputenc}
  \usepackage{textcomp} % provide euro and other symbols
\else % if luatex or xetex
  \usepackage{unicode-math} % this also loads fontspec
  \defaultfontfeatures{Scale=MatchLowercase}
  \defaultfontfeatures[\rmfamily]{Ligatures=TeX,Scale=1}
\fi
\usepackage{lmodern}
\ifPDFTeX\else
  % xetex/luatex font selection
\fi
% Use upquote if available, for straight quotes in verbatim environments
\IfFileExists{upquote.sty}{\usepackage{upquote}}{}
\IfFileExists{microtype.sty}{% use microtype if available
  \usepackage[]{microtype}
  \UseMicrotypeSet[protrusion]{basicmath} % disable protrusion for tt fonts
}{}
\makeatletter
\@ifundefined{KOMAClassName}{% if non-KOMA class
  \IfFileExists{parskip.sty}{%
    \usepackage{parskip}
  }{% else
    \setlength{\parindent}{0pt}
    \setlength{\parskip}{6pt plus 2pt minus 1pt}}
}{% if KOMA class
  \KOMAoptions{parskip=half}}
\makeatother
% Make \paragraph and \subparagraph free-standing
\makeatletter
\ifx\paragraph\undefined\else
  \let\oldparagraph\paragraph
  \renewcommand{\paragraph}{
    \@ifstar
      \xxxParagraphStar
      \xxxParagraphNoStar
  }
  \newcommand{\xxxParagraphStar}[1]{\oldparagraph*{#1}\mbox{}}
  \newcommand{\xxxParagraphNoStar}[1]{\oldparagraph{#1}\mbox{}}
\fi
\ifx\subparagraph\undefined\else
  \let\oldsubparagraph\subparagraph
  \renewcommand{\subparagraph}{
    \@ifstar
      \xxxSubParagraphStar
      \xxxSubParagraphNoStar
  }
  \newcommand{\xxxSubParagraphStar}[1]{\oldsubparagraph*{#1}\mbox{}}
  \newcommand{\xxxSubParagraphNoStar}[1]{\oldsubparagraph{#1}\mbox{}}
\fi
\makeatother

\usepackage{color}
\usepackage{fancyvrb}
\newcommand{\VerbBar}{|}
\newcommand{\VERB}{\Verb[commandchars=\\\{\}]}
\DefineVerbatimEnvironment{Highlighting}{Verbatim}{commandchars=\\\{\}}
% Add ',fontsize=\small' for more characters per line
\usepackage{framed}
\definecolor{shadecolor}{RGB}{241,243,245}
\newenvironment{Shaded}{\begin{snugshade}}{\end{snugshade}}
\newcommand{\AlertTok}[1]{\textcolor[rgb]{0.68,0.00,0.00}{#1}}
\newcommand{\AnnotationTok}[1]{\textcolor[rgb]{0.37,0.37,0.37}{#1}}
\newcommand{\AttributeTok}[1]{\textcolor[rgb]{0.40,0.45,0.13}{#1}}
\newcommand{\BaseNTok}[1]{\textcolor[rgb]{0.68,0.00,0.00}{#1}}
\newcommand{\BuiltInTok}[1]{\textcolor[rgb]{0.00,0.23,0.31}{#1}}
\newcommand{\CharTok}[1]{\textcolor[rgb]{0.13,0.47,0.30}{#1}}
\newcommand{\CommentTok}[1]{\textcolor[rgb]{0.37,0.37,0.37}{#1}}
\newcommand{\CommentVarTok}[1]{\textcolor[rgb]{0.37,0.37,0.37}{\textit{#1}}}
\newcommand{\ConstantTok}[1]{\textcolor[rgb]{0.56,0.35,0.01}{#1}}
\newcommand{\ControlFlowTok}[1]{\textcolor[rgb]{0.00,0.23,0.31}{\textbf{#1}}}
\newcommand{\DataTypeTok}[1]{\textcolor[rgb]{0.68,0.00,0.00}{#1}}
\newcommand{\DecValTok}[1]{\textcolor[rgb]{0.68,0.00,0.00}{#1}}
\newcommand{\DocumentationTok}[1]{\textcolor[rgb]{0.37,0.37,0.37}{\textit{#1}}}
\newcommand{\ErrorTok}[1]{\textcolor[rgb]{0.68,0.00,0.00}{#1}}
\newcommand{\ExtensionTok}[1]{\textcolor[rgb]{0.00,0.23,0.31}{#1}}
\newcommand{\FloatTok}[1]{\textcolor[rgb]{0.68,0.00,0.00}{#1}}
\newcommand{\FunctionTok}[1]{\textcolor[rgb]{0.28,0.35,0.67}{#1}}
\newcommand{\ImportTok}[1]{\textcolor[rgb]{0.00,0.46,0.62}{#1}}
\newcommand{\InformationTok}[1]{\textcolor[rgb]{0.37,0.37,0.37}{#1}}
\newcommand{\KeywordTok}[1]{\textcolor[rgb]{0.00,0.23,0.31}{\textbf{#1}}}
\newcommand{\NormalTok}[1]{\textcolor[rgb]{0.00,0.23,0.31}{#1}}
\newcommand{\OperatorTok}[1]{\textcolor[rgb]{0.37,0.37,0.37}{#1}}
\newcommand{\OtherTok}[1]{\textcolor[rgb]{0.00,0.23,0.31}{#1}}
\newcommand{\PreprocessorTok}[1]{\textcolor[rgb]{0.68,0.00,0.00}{#1}}
\newcommand{\RegionMarkerTok}[1]{\textcolor[rgb]{0.00,0.23,0.31}{#1}}
\newcommand{\SpecialCharTok}[1]{\textcolor[rgb]{0.37,0.37,0.37}{#1}}
\newcommand{\SpecialStringTok}[1]{\textcolor[rgb]{0.13,0.47,0.30}{#1}}
\newcommand{\StringTok}[1]{\textcolor[rgb]{0.13,0.47,0.30}{#1}}
\newcommand{\VariableTok}[1]{\textcolor[rgb]{0.07,0.07,0.07}{#1}}
\newcommand{\VerbatimStringTok}[1]{\textcolor[rgb]{0.13,0.47,0.30}{#1}}
\newcommand{\WarningTok}[1]{\textcolor[rgb]{0.37,0.37,0.37}{\textit{#1}}}

\usepackage{longtable,booktabs,array}
\usepackage{calc} % for calculating minipage widths
% Correct order of tables after \paragraph or \subparagraph
\usepackage{etoolbox}
\makeatletter
\patchcmd\longtable{\par}{\if@noskipsec\mbox{}\fi\par}{}{}
\makeatother
% Allow footnotes in longtable head/foot
\IfFileExists{footnotehyper.sty}{\usepackage{footnotehyper}}{\usepackage{footnote}}
\makesavenoteenv{longtable}
\usepackage{graphicx}
\makeatletter
\newsavebox\pandoc@box
\newcommand*\pandocbounded[1]{% scales image to fit in text height/width
  \sbox\pandoc@box{#1}%
  \Gscale@div\@tempa{\textheight}{\dimexpr\ht\pandoc@box+\dp\pandoc@box\relax}%
  \Gscale@div\@tempb{\linewidth}{\wd\pandoc@box}%
  \ifdim\@tempb\p@<\@tempa\p@\let\@tempa\@tempb\fi% select the smaller of both
  \ifdim\@tempa\p@<\p@\scalebox{\@tempa}{\usebox\pandoc@box}%
  \else\usebox{\pandoc@box}%
  \fi%
}
% Set default figure placement to htbp
\def\fps@figure{htbp}
\makeatother





\setlength{\emergencystretch}{3em} % prevent overfull lines

\providecommand{\tightlist}{%
  \setlength{\itemsep}{0pt}\setlength{\parskip}{0pt}}



 


\KOMAoption{captions}{tableheading}
\makeatletter
\@ifpackageloaded{tcolorbox}{}{\usepackage[skins,breakable]{tcolorbox}}
\@ifpackageloaded{fontawesome5}{}{\usepackage{fontawesome5}}
\definecolor{quarto-callout-color}{HTML}{909090}
\definecolor{quarto-callout-note-color}{HTML}{0758E5}
\definecolor{quarto-callout-important-color}{HTML}{CC1914}
\definecolor{quarto-callout-warning-color}{HTML}{EB9113}
\definecolor{quarto-callout-tip-color}{HTML}{00A047}
\definecolor{quarto-callout-caution-color}{HTML}{FC5300}
\definecolor{quarto-callout-color-frame}{HTML}{acacac}
\definecolor{quarto-callout-note-color-frame}{HTML}{4582ec}
\definecolor{quarto-callout-important-color-frame}{HTML}{d9534f}
\definecolor{quarto-callout-warning-color-frame}{HTML}{f0ad4e}
\definecolor{quarto-callout-tip-color-frame}{HTML}{02b875}
\definecolor{quarto-callout-caution-color-frame}{HTML}{fd7e14}
\makeatother
\makeatletter
\@ifpackageloaded{bookmark}{}{\usepackage{bookmark}}
\makeatother
\makeatletter
\@ifpackageloaded{caption}{}{\usepackage{caption}}
\AtBeginDocument{%
\ifdefined\contentsname
  \renewcommand*\contentsname{Table of contents}
\else
  \newcommand\contentsname{Table of contents}
\fi
\ifdefined\listfigurename
  \renewcommand*\listfigurename{List of Figures}
\else
  \newcommand\listfigurename{List of Figures}
\fi
\ifdefined\listtablename
  \renewcommand*\listtablename{List of Tables}
\else
  \newcommand\listtablename{List of Tables}
\fi
\ifdefined\figurename
  \renewcommand*\figurename{Figure}
\else
  \newcommand\figurename{Figure}
\fi
\ifdefined\tablename
  \renewcommand*\tablename{Table}
\else
  \newcommand\tablename{Table}
\fi
}
\@ifpackageloaded{float}{}{\usepackage{float}}
\floatstyle{ruled}
\@ifundefined{c@chapter}{\newfloat{codelisting}{h}{lop}}{\newfloat{codelisting}{h}{lop}[chapter]}
\floatname{codelisting}{Listing}
\newcommand*\listoflistings{\listof{codelisting}{List of Listings}}
\makeatother
\makeatletter
\makeatother
\makeatletter
\@ifpackageloaded{caption}{}{\usepackage{caption}}
\@ifpackageloaded{subcaption}{}{\usepackage{subcaption}}
\makeatother
\usepackage{bookmark}
\IfFileExists{xurl.sty}{\usepackage{xurl}}{} % add URL line breaks if available
\urlstyle{same}
\hypersetup{
  pdftitle={Computational Optimization \& Applications},
  pdfauthor={Siju Swamy},
  colorlinks=true,
  linkcolor={blue},
  filecolor={Maroon},
  citecolor={Blue},
  urlcolor={Blue},
  pdfcreator={LaTeX via pandoc}}


\title{Computational Optimization \& Applications}
\author{Siju Swamy}
\date{2025-11-17}
\begin{document}
\maketitle

\renewcommand*\contentsname{Table of contents}
{
\hypersetup{linkcolor=}
\setcounter{tocdepth}{2}
\tableofcontents
}

\bookmarksetup{startatroot}

\chapter*{🎯 Course Vision \& Context}\label{course-vision-context}
\addcontentsline{toc}{chapter}{🎯 Course Vision \& Context}

\markboth{🎯 Course Vision \& Context}{🎯 Course Vision \& Context}

\subsection*{The Optimization Revolution in Computational
Sciences}\label{the-optimization-revolution-in-computational-sciences}
\addcontentsline{toc}{subsection}{The Optimization Revolution in
Computational Sciences}

In an era dominated by Artificial Intelligence, Machine Learning, and
Data Science, \emph{optimization forms the fundamental backbone} that
powers intelligent decision-making systems. From recommending your next
movie to orchestrating global supply chains, from training deep neural
networks to scheduling autonomous vehicles---optimization algorithms are
the invisible engines driving technological progress.

This course positions you at the intersection of \emph{mathematical
theory} and \emph{computational practice}, equipping you with both the
conceptual understanding and hands-on skills to design, implement, and
deploy optimization solutions for real-world challenges.

\section*{Course Syllabus (48 Hours / 12
Weeks)}\label{course-syllabus-48-hours-12-weeks}
\addcontentsline{toc}{section}{Course Syllabus (48 Hours / 12 Weeks)}

\markright{Course Syllabus (48 Hours / 12 Weeks)}

\subsection*{Course Overview}\label{course-overview}
\addcontentsline{toc}{subsection}{Course Overview}

\textbf{Course Code:} 20MAT382

\textbf{Course Name:} Computational Optimization and Applications

\textbf{Duration:} 12 Weeks (4 hours/week)

\textbf{Credits:} 4

\textbf{Total Marks:} 150 (Internal: 70 + External: 80)

\subsubsection*{Intensive Learning
Approach}\label{intensive-learning-approach}
\addcontentsline{toc}{subsubsection}{Intensive Learning Approach}

Accelerated project-based curriculum focusing on core optimization
concepts with immediate practical application through integrated
micro-projects.

\subsection*{Course Objectives}\label{course-objectives}
\addcontentsline{toc}{subsection}{Course Objectives}

\begin{longtable}[]{@{}
  >{\raggedright\arraybackslash}p{(\linewidth - 2\tabcolsep) * \real{0.2400}}
  >{\raggedright\arraybackslash}p{(\linewidth - 2\tabcolsep) * \real{0.7600}}@{}}
\toprule\noalign{}
\begin{minipage}[b]{\linewidth}\raggedright
S.No
\end{minipage} & \begin{minipage}[b]{\linewidth}\raggedright
COURSE OBJECTIVES
\end{minipage} \\
\midrule\noalign{}
\endhead
\bottomrule\noalign{}
\endlastfoot
1 & To gain a comprehensive understanding of optimization concepts and
their real-world relevance, emphasizing Python as a practical tool for
optimization. \\
2 & To develop proficiency in Python for optimization, including
formulating and solving linear programming problems, implementing
nonlinear optimization and analysing optimization solutions. \\
3 & To acquire skills in project planning and optimization techniques
using Python. \\
4 & To master optimization techniques in the context of machine
learning, including Gradient Descent, Stochastic Gradient Descent, and
various optimization algorithms, all implemented in Python. \\
5 & To apply optimization knowledge and Python skills to solve
combinatorial and graph-based problems, while also considering the
ethical aspects of optimization in engineering, logistics, and
decision-making. \\
\end{longtable}

\section*{Course Outcomes}\label{course-outcomes}
\addcontentsline{toc}{section}{Course Outcomes}

\markright{Course Outcomes}

At the end of the course students will be able to:

\begin{longtable}[]{@{}
  >{\raggedright\arraybackslash}p{(\linewidth - 4\tabcolsep) * \real{0.0811}}
  >{\raggedright\arraybackslash}p{(\linewidth - 4\tabcolsep) * \real{0.6306}}
  >{\raggedright\arraybackslash}p{(\linewidth - 4\tabcolsep) * \real{0.2883}}@{}}
\toprule\noalign{}
\begin{minipage}[b]{\linewidth}\raggedright
CO Code
\end{minipage} & \begin{minipage}[b]{\linewidth}\raggedright
COURSE OUTCOMES
\end{minipage} & \begin{minipage}[b]{\linewidth}\raggedright
REVISED BLOOM'S TAXONOMY LEVEL
\end{minipage} \\
\midrule\noalign{}
\endhead
\bottomrule\noalign{}
\endlastfoot
CO1 & Demonstrate a thorough understanding of optimization concepts,
problem types, and their real-world applications. & 3 \\
CO2 & Formulate and solve linear programming problems using Python. &
3 \\
CO3 & Implement nonlinear optimization algorithms, and analyse
optimization solutions using Python. & 2 \\
CO4 & Develop practical project planning skills and proficiency in
applying heuristic algorithms to real-world scenarios. & 2 \\
CO5 & Demonstrate mastery of optimization in Machine Learning, with the
ability to apply Gradient Descent, Stochastic Gradient Descent. & 3 \\
\end{longtable}

\subsubsection*{Core Competencies}\label{core-competencies}
\addcontentsline{toc}{subsubsection}{Core Competencies}

\begin{itemize}
\tightlist
\item
  Formulate real-world problems as mathematical optimization models
\item
  Implement optimization algorithms in Python using industry tools
\item
  Analyze and validate optimization solutions
\item
  Develop end-to-end optimization systems for practical applications
\end{itemize}

\subsection*{Assessment Plan (70 Marks
Internal)}\label{assessment-plan-70-marks-internal}
\addcontentsline{toc}{subsection}{Assessment Plan (70 Marks Internal)}

\subsubsection*{Continuous Evaluation (70
Marks)}\label{continuous-evaluation-70-marks}
\addcontentsline{toc}{subsubsection}{Continuous Evaluation (70 Marks)}

\textbf{A. Theory Components (20 Marks)}

\begin{itemize}
\tightlist
\item
  \textbf{Internal Exam 1}: 10 marks (Week 6)
\item
  \textbf{Internal Exam 2}: 10 marks (Week 12)
\end{itemize}

\textbf{B. Practical Components (50 Marks)}

\begin{itemize}
\tightlist
\item
  \textbf{Assignments/Micro Projects}: 15 marks (3 projects × 5 marks
  each)
\item
  \textbf{Lab Exams}: 10 marks (2 exams × 5 marks each)
\item
  \textbf{Day-to-Day Lab Work}: 15 marks
\item
  \textbf{Attendance}: 10 marks
\end{itemize}

\subsubsection*{External Examination (80
Marks)}\label{external-examination-80-marks}
\addcontentsline{toc}{subsubsection}{External Examination (80 Marks)}

\begin{itemize}
\tightlist
\item
  \textbf{End Semester Theory Exam}: 80 marks
\end{itemize}

\subsection*{🗓️ 12-Week Delivery Plan (4
Hours/Week)}\label{week-delivery-plan-4-hoursweek}
\addcontentsline{toc}{subsection}{🗓️ 12-Week Delivery Plan (4
Hours/Week)}

\subsubsection*{Phase 1: Foundation \& Linear Methods (Weeks
1-4)}\label{phase-1-foundation-linear-methods-weeks-1-4}
\addcontentsline{toc}{subsubsection}{Phase 1: Foundation \& Linear
Methods (Weeks 1-4)}

\textbf{Week 1: Optimization Fundamentals \& Python Setup} (4 hours)

\begin{itemize}
\tightlist
\item
  \textbf{Theory (2h)}: Optimization concepts, problem classification,
  LP formulation
\item
  \textbf{Lab (2h)}: \texttt{Python} environment setup, \texttt{PuLP}
  introduction
\item
  \textbf{Lab Work}: Basic LP implementation (1 mark)
\item
  \textbf{Micro-Project 1 Launch}: Campus facility location problem
\end{itemize}

\textbf{Week 2: Linear Programming \& Solution Methods} (4 hours)

\begin{itemize}
\tightlist
\item
  \textbf{Theory (2h)}: Graphical method, Simplex algorithm
\item
  \textbf{Lab (2h)}: \texttt{PuLP} implementation, constraint handling
\item
  \textbf{Lab Work}: Complex constraint implementation (1 mark)
\item
  \textbf{Attendance}: Week 1-2 (2 marks)
\end{itemize}

\textbf{Week 3: Advanced LP \& Real Applications} (4 hours)

\begin{itemize}
\tightlist
\item
  \textbf{Theory (1h)}: Sensitivity analysis, duality
\item
  \textbf{Lab (3h)}: Transportation problems, case studies
\item
  \textbf{Lab Work}: Transportation problem solution (1 mark)
\item
  \textbf{Micro-Project 1 Due}: Submission (5 marks)
\end{itemize}

\textbf{Week 4: Nonlinear Optimization Foundations} (4 hours)

\begin{itemize}
\tightlist
\item
  \textbf{Theory (2h)}: Unconstrained optimization, Golden Section
\item
  \textbf{Lab (2h)}: \texttt{SciPy} optimization, function minimization
\item
  \textbf{Lab Work}: Nonlinear solver implementation (1 mark)
\item
  \textbf{Attendance}: Week 3-4 (2 marks)
\end{itemize}

\subsubsection*{Phase 2: Constrained \& Combinatorial Methods (Weeks
5-8)}\label{phase-2-constrained-combinatorial-methods-weeks-5-8}
\addcontentsline{toc}{subsubsection}{Phase 2: Constrained \&
Combinatorial Methods (Weeks 5-8)}

\textbf{Week 5: Constrained Optimization} (4 hours)

\begin{itemize}
\tightlist
\item
  \textbf{Theory (2h)}: KKT conditions, constraint handling
\item
  \textbf{Lab (2h)}: Constrained NLP implementation
\item
  \textbf{Lab Work}: KKT condition implementation (1 mark)
\item
  \textbf{Micro-Project 2 Launch}: Nonlinear cost optimization
\item
  \textbf{Lab Exam 1}: Basic LP/NLP implementation (5 marks)
\end{itemize}

\textbf{Week 6: Project Planning \& Heuristics} (4 hours)

\begin{itemize}
\tightlist
\item
  \textbf{Theory (1h)}: CPM/PERT fundamentals
\item
  \textbf{Lab (3h)}: Project scheduling, greedy algorithms
\item
  \textbf{Internal Exam 1}: Theory assessment (10 marks)
\item
  \textbf{Lab Work}: Project scheduling implementation (1 mark)
\end{itemize}

\textbf{Week 7: Graph Algorithms I} (4 hours)

\begin{itemize}
\tightlist
\item
  \textbf{Theory (1h)}: Graph theory, shortest path concepts
\item
  \textbf{Lab (3h)}: NetworkX implementation, Dijkstra's algorithm
\item
  \textbf{Lab Work}: Shortest path implementation (1 mark)
\item
  \textbf{Micro-Project 2 Due}: Submission (5 marks)
\item
  \textbf{Attendance}: Week 5-7 (2 marks)
\end{itemize}

\textbf{Week 8: Graph Algorithms II} (4 hours)

\begin{itemize}
\tightlist
\item
  \textbf{Theory (1h)}: MST, network flows, TSP overview
\item
  \textbf{Lab (3h)}: Advanced graph algorithms
\item
  \textbf{Lab Work}: MST implementation (1 mark)
\item
  \textbf{Micro-Project 3 Launch}: Routing optimization
\end{itemize}

\subsubsection*{Phase 3: Advanced Applications \& Integration (Weeks
9-12)}\label{phase-3-advanced-applications-integration-weeks-9-12}
\addcontentsline{toc}{subsubsection}{Phase 3: Advanced Applications \&
Integration (Weeks 9-12)}

\textbf{Week 9: Machine Learning Optimization I} (4 hours)

\begin{itemize}
\tightlist
\item
  \textbf{Theory (2h)}: Gradient Descent, SGD, optimization in ML
\item
  \textbf{Lab (2h)}: Basic GD implementation
\item
  \textbf{Lab Work}: Gradient descent implementation (1 mark)
\item
  \textbf{Attendance}: Week 8-9 (2 marks)
\end{itemize}

\textbf{Week 10: Machine Learning Optimization II} (4 hours)

\begin{itemize}
\tightlist
\item
  \textbf{Theory (1h)}: Advanced optimizers, neural networks
\item
  \textbf{Lab (3h)}: TensorFlow/PyTorch optimization
\item
  \textbf{Lab Work}: Advanced optimizer implementation (1 mark)
\item
  \textbf{Lab Exam 2}: Graph and ML optimization (5 marks)
\end{itemize}

\textbf{Week 11: Integrated Applications} (4 hours)

\begin{itemize}
\tightlist
\item
  \textbf{Workshop (4h)}: Comprehensive system implementation
\item
  \textbf{Lab Work}: Integrated system development (2 marks)
\item
  \textbf{Micro-Project 3 Due}: Submission (5 marks)
\end{itemize}

\textbf{Week 12: Review \& Final Assessment} (4 hours)

\begin{itemize}
\tightlist
\item
  \textbf{Internal Exam 2}: Theory assessment (10 marks)
\item
  \textbf{Course Review}: Comprehensive concepts revision
\item
  \textbf{Lab Work}: Final implementation polish (1 mark)
\item
  \textbf{Attendance}: Week 10-12 (2 marks)
\end{itemize}

\subsection*{Thematic Project: Campus City Supply
Chain}\label{thematic-project-campus-city-supply-chain}
\addcontentsline{toc}{subsection}{Thematic Project: Campus City Supply
Chain}

\subsubsection*{Micro-Projects (15 Marks
Total)}\label{micro-projects-15-marks-total}
\addcontentsline{toc}{subsubsection}{Micro-Projects (15 Marks Total)}

\textbf{Micro-Project 1: Basic LP Implementation} (5 marks)

\begin{itemize}
\tightlist
\item
  \textbf{Timeline}: Week 1-3
\item
  \textbf{Scope}: 6 facilities, 3 warehouses, linear costs
\item
  \textbf{Assessment}: Model correctness (2), Code quality (2),
  Documentation (1)
\end{itemize}

\textbf{Micro-Project 2: Nonlinear Optimization} (5 marks)

\begin{itemize}
\tightlist
\item
  \textbf{Timeline}: Week 4-7
\item
  \textbf{Scope}: Enhanced cost models, KKT conditions
\item
  \textbf{Assessment}: Algorithm implementation (2), Analysis (2),
  Validation (1)
\end{itemize}

\textbf{Micro-Project 3: Graph \& Network Optimization} (5 marks)

\begin{itemize}
\tightlist
\item
  \textbf{Timeline}: Week 8-11
\item
  \textbf{Scope}: Routing, shortest paths, resource allocation
\item
  \textbf{Assessment}: System design (2), Performance (2), Documentation
  (1)
\end{itemize}

\subsection*{Detailed Mark
Distribution}\label{detailed-mark-distribution}
\addcontentsline{toc}{subsection}{Detailed Mark Distribution}

\subsubsection*{Day-to-Day Lab Work (15
Marks)}\label{day-to-day-lab-work-15-marks}
\addcontentsline{toc}{subsubsection}{Day-to-Day Lab Work (15 Marks)}

\begin{itemize}
\tightlist
\item
  \textbf{Weekly Implementation Tasks}: 12 marks (1 mark × 12 weeks)
\item
  \textbf{Integrated System Development}: 3 marks (Week 11)
\end{itemize}

\subsubsection*{Attendance (10 Marks)}\label{attendance-10-marks}
\addcontentsline{toc}{subsubsection}{Attendance (10 Marks)}

\begin{itemize}
\tightlist
\item
  \textbf{Weekly Attendance}: 2 marks per 3-week block
\item
  \textbf{Full Attendance Bonus}: 2 marks for 100\% attendance
\end{itemize}

\subsubsection*{Lab Exams (10 Marks)}\label{lab-exams-10-marks}
\addcontentsline{toc}{subsubsection}{Lab Exams (10 Marks)}

\begin{itemize}
\tightlist
\item
  \textbf{Lab Exam 1} (Week 5): Basic LP/NLP implementation (5 marks)
\item
  \textbf{Lab Exam 2} (Week 10): Graph and ML optimization (5 marks)
\end{itemize}

\subsubsection*{Internal Exams (20
Marks)}\label{internal-exams-20-marks}
\addcontentsline{toc}{subsubsection}{Internal Exams (20 Marks)}

\begin{itemize}
\tightlist
\item
  \textbf{Internal Exam 1} (Week 6): Modules 1-2 theory (10 marks)
\item
  \textbf{Internal Exam 2} (Week 11): Modules 3-4 theory (10 marks)
\end{itemize}

\subsection*{Learning Outcomes Mapping}\label{learning-outcomes-mapping}
\addcontentsline{toc}{subsection}{Learning Outcomes Mapping}

\subsubsection*{Theory Outcomes (Internal Exams +
External)}\label{theory-outcomes-internal-exams-external}
\addcontentsline{toc}{subsubsection}{Theory Outcomes (Internal Exams +
External)}

\begin{itemize}
\tightlist
\item
  Formulate optimization problems mathematically
\item
  Understand algorithm properties and convergence
\item
  Analyze problem structures and solution methods
\end{itemize}

\subsubsection*{Practical Outcomes (Lab Work +
Projects)}\label{practical-outcomes-lab-work-projects}
\addcontentsline{toc}{subsubsection}{Practical Outcomes (Lab Work +
Projects)}

\begin{itemize}
\tightlist
\item
  Implement optimization algorithms in Python
\item
  Develop end-to-end optimization systems
\item
  Validate and analyze optimization results
\item
  Create professional documentation and visualizations
\end{itemize}

\subsection*{Grading Rubrics}\label{grading-rubrics}
\addcontentsline{toc}{subsection}{Grading Rubrics}

\subsubsection*{Micro-Projects (5 marks
each)}\label{micro-projects-5-marks-each}
\addcontentsline{toc}{subsubsection}{Micro-Projects (5 marks each)}

\begin{itemize}
\tightlist
\item
  \textbf{Excellent (5)}: Flawless implementation with advanced features
\item
  \textbf{Very Good (4)}: Correct implementation with good documentation
\item
  \textbf{Good (3)}: Basic functionality with minor issues
\item
  \textbf{Satisfactory (2)}: Meets minimum requirements
\item
  \textbf{Poor (1)}: Significant functionality missing
\end{itemize}

\subsubsection*{Lab Work (Weekly 1 mark)}\label{lab-work-weekly-1-mark}
\addcontentsline{toc}{subsubsection}{Lab Work (Weekly 1 mark)}

\begin{itemize}
\tightlist
\item
  \textbf{Complete (1)}: Task fully implemented and demonstrated
\item
  \textbf{Partial (0.5)}: Basic implementation with issues
\item
  \textbf{Incomplete (0)}: Task not attempted or completely
  non-functional
\end{itemize}

\subsubsection*{Lab Exams (5 marks each)}\label{lab-exams-5-marks-each}
\addcontentsline{toc}{subsubsection}{Lab Exams (5 marks each)}

\begin{itemize}
\tightlist
\item
  \textbf{Algorithm Implementation}: 2 marks
\item
  \textbf{Problem Solving}: 2 marks
\item
  \textbf{Code Quality}: 1 mark
\end{itemize}

\subsection*{Success Strategy}\label{success-strategy}
\addcontentsline{toc}{subsection}{Success Strategy}

\subsubsection*{Maximizing Internal
Marks}\label{maximizing-internal-marks}
\addcontentsline{toc}{subsubsection}{Maximizing Internal Marks}

\begin{itemize}
\tightlist
\item
  \textbf{Consistent Attendance}: 10 marks easily achievable
\item
  \textbf{Regular Lab Work}: 15 marks through weekly completion
\item
  \textbf{Quality Projects}: 15 marks with careful implementation
\item
  \textbf{Lab Exam Preparation}: 10 marks with practice
\item
  \textbf{Internal Exam Focus}: 20 marks through concept mastery
\end{itemize}

\subsubsection*{External Exam Preparation (80
Marks)}\label{external-exam-preparation-80-marks}
\addcontentsline{toc}{subsubsection}{External Exam Preparation (80
Marks)}

\begin{itemize}
\tightlist
\item
  Comprehensive theory coverage from all modules
\item
  Problem-solving practice with various optimization types
\item
  Mathematical formulation skills
\item
  Algorithm analysis and comparison
\end{itemize}

\subsection*{Weekly Preparation Guide}\label{weekly-preparation-guide}
\addcontentsline{toc}{subsection}{Weekly Preparation Guide}

\subsubsection*{Before Each Week}\label{before-each-week}
\addcontentsline{toc}{subsubsection}{Before Each Week}

\begin{itemize}
\tightlist
\item
  Review weekly objectives and deliverables
\item
  Prepare development environment
\item
  Read theoretical concepts in advance
\end{itemize}

\subsubsection*{During Each Week}\label{during-each-week}
\addcontentsline{toc}{subsubsection}{During Each Week}

\begin{itemize}
\tightlist
\item
  Attend all sessions (critical for attendance marks)
\item
  Complete lab work during sessions
\item
  Start micro-projects early
\item
  Seek clarification immediately
\end{itemize}

\subsubsection*{After Each Week}\label{after-each-week}
\addcontentsline{toc}{subsubsection}{After Each Week}

\begin{itemize}
\tightlist
\item
  Submit all lab work promptly
\item
  Review concepts for internal exams
\item
  Prepare for upcoming assessments
\item
  Maintain code repository
\end{itemize}

\begin{center}\rule{0.5\linewidth}{0.5pt}\end{center}

\emph{This assessment-focused syllabus ensures students can maximize
their 70 internal marks through consistent performance while preparing
comprehensively for the 80-mark external examination. The structured
approach balances theoretical understanding with practical
implementation skills.}

\bookmarksetup{startatroot}

\chapter{Introduction to Computational Optimization and
Applications}\label{introduction-to-computational-optimization-and-applications}

\begin{verbatim}
🚀 Welcome to Computational Optimization & Applications
📚 Where Mathematics Meets Computational Intelligence
🌉 Bridging Theory and Practice in the AI Revolution
\end{verbatim}

\begin{center}\rule{0.5\linewidth}{0.5pt}\end{center}

\section{Why Optimization Matters Now More Than
Ever}\label{why-optimization-matters-now-more-than-ever}

The exponential growth in data complexity and computational requirements
has transformed optimization from a theoretical discipline to an
essential toolkit for every computer scientist and data professional.
Consider these real-world contexts:

\begin{itemize}
\tightlist
\item
  \textbf{AI Systems}: Training neural networks is essentially an
  optimization process (Gradient Descent)
\item
  \textbf{Operations Research}: Logistics, scheduling, and resource
  allocation drive billion-dollar efficiencies\\
\item
  \textbf{Data Science}: Model selection, hyperparameter tuning, and
  feature engineering are optimization problems
\item
  \textbf{Autonomous Systems}: Path planning, control systems, and
  decision-making rely on optimization algorithms
\item
  \textbf{Quantum Computing}: Many quantum algorithms are designed to
  solve optimization problems more efficiently
\end{itemize}

\section{📋 Programme Objectives \& Learning
Outcomes}\label{programme-objectives-learning-outcomes}

\subsection{Core Educational Mission}\label{core-educational-mission}

This minor programme is designed to bridge the critical gap between
theoretical optimization mathematics and practical computational
implementation. Upon successful completion, you will be able to:

\begin{longtable}[]{@{}
  >{\raggedright\arraybackslash}p{(\linewidth - 2\tabcolsep) * \real{0.3529}}
  >{\raggedright\arraybackslash}p{(\linewidth - 2\tabcolsep) * \real{0.6471}}@{}}
\toprule\noalign{}
\begin{minipage}[b]{\linewidth}\raggedright
\textbf{Domain}
\end{minipage} & \begin{minipage}[b]{\linewidth}\raggedright
\textbf{Learning Outcomes}
\end{minipage} \\
\midrule\noalign{}
\endhead
\bottomrule\noalign{}
\endlastfoot
\textbf{Theoretical Foundation} & • Formulate real-world problems as
mathematical optimization models• Understand optimality conditions and
convergence properties• Analyze problem structures to select appropriate
solution methods \\
\textbf{Computational Skills} & • Implement classical and modern
optimization algorithms in Python• Utilize industry-standard
optimization libraries and frameworks• Develop end-to-end optimization
pipelines for practical applications \\
\textbf{AI/ML Integration} & • Understand optimization's role in
training machine learning models• Implement gradient-based methods for
neural network optimization• Apply optimization to hyperparameter tuning
and model selection \\
\textbf{Problem-Solving} & • Design optimization solutions for complex,
multi-objective problems• Evaluate solution quality and algorithm
performance• Communicate optimization insights to technical and
non-technical stakeholders \\
\end{longtable}

\subsection{The ``Smart City Logistics'' Experiential
Thread}\label{the-smart-city-logistics-experiential-thread}

Throughout this course, we'll employ a \emph{continuous practical
thread}: optimizing logistics and operations for a smart city ecosystem.
This narrative provides:

\begin{itemize}
\tightlist
\item
  \textbf{Real-world context} for theoretical concepts
\item
  \textbf{Progressive complexity} as we advance through modules
\item
  \textbf{Portfolio-building} implementation experience
\item
  \textbf{Industry-relevant} problem-solving skills
\end{itemize}

\textbf{Smart City Optimization Journey:}

\begin{itemize}
\tightlist
\item
  \textbf{Module 1}: Warehouse Location (Linear Programming)
\item
  \textbf{Module 2}: Fuel Cost Optimization (Nonlinear Programming)
\item
  \textbf{Module 3}: Delivery Routing (Graph Algorithms)
\item
  \textbf{Module 4}: Dynamic Scheduling (Heuristic Methods)
\item
  \textbf{Module 5}: Demand Prediction (ML Integration)
\end{itemize}

\section{Course Roadmap \& Syllabus
Integration}\label{course-roadmap-syllabus-integration}

\subsection{Module Progression: From Foundations to
Frontiers}\label{module-progression-from-foundations-to-frontiers}

Our journey through computational optimization is strategically
sequenced to build from fundamental principles to advanced applications:

\subsubsection{Foundation Phase: Mathematical
Underpinnings}\label{foundation-phase-mathematical-underpinnings}

\begin{itemize}
\tightlist
\item
  \textbf{Module I}: Linear Programming \& Formulation Skills
\item
  \textbf{Module II}: Nonlinear Optimization \& Constraint Handling
\end{itemize}

\subsubsection{Advanced Phase: Algorithmic
Thinking}\label{advanced-phase-algorithmic-thinking}

\begin{itemize}
\tightlist
\item
  \textbf{Module III}: Project Planning \& Heuristic Methods
\item
  \textbf{Module IV}: Combinatorial \& Graph Optimization
\end{itemize}

\subsubsection{Integration Phase: AI/ML
Applications}\label{integration-phase-aiml-applications}

\begin{itemize}
\tightlist
\item
  \textbf{Module V}: Gradient Methods \& Machine Learning Optimization
\end{itemize}

\subsection{Assessment Strategy: Theory Meets
Practice}\label{assessment-strategy-theory-meets-practice}

To ensure comprehensive understanding and skill development, assessment
integrates both theoretical knowledge and practical implementation:

\begin{itemize}
\tightlist
\item
  \textbf{Series Examinations}: Test conceptual understanding and
  problem formulation
\item
  \textbf{Practical Assignments/ Micro Project}: Evaluate implementation
  skills and computational thinking\\
\item
  \textbf{Final Project}: Assess integrated problem-solving and solution
  design
\item
  \textbf{Continuous Evaluation}: Monitor progress through
  micro-projects and code reviews
\end{itemize}

\section{Optimization in the AI/ML
Ecosystem}\label{optimization-in-the-aiml-ecosystem}

\subsection{The Central Role in Machine
Learning}\label{the-central-role-in-machine-learning}

Optimization isn't just adjacent to machine learning---\textbf{it is
machine learning}. The entire process of training machine learning
models revolves around optimization principles:

\begin{itemize}
\tightlist
\item
  \textbf{Loss Minimization}: Finding model parameters that minimize
  prediction error
\item
  \textbf{Convergence Analysis}: Understanding when and how algorithms
  reach optimal solutions
\item
  \textbf{Regularization}: Balancing model complexity with performance
  through constrained optimization
\item
  \textbf{Hyperparameter Tuning}: Optimizing the optimization process
  itself
\end{itemize}

\subsection{Emerging Trends \& Future
Directions}\label{emerging-trends-future-directions}

The field of optimization is rapidly evolving, driven by advances in:

\begin{itemize}
\tightlist
\item
  \textbf{Large-Scale Optimization}: Methods for billion-parameter
  models in deep learning
\item
  \textbf{Automated Optimization}: AutoML and neural architecture search
\item
  \textbf{Quantum Optimization}: Quantum annealing and hybrid
  quantum-classical algorithms\\
\item
  \textbf{Federated Optimization}: Privacy-preserving distributed
  learning
\item
  \textbf{Multi-Objective Optimization}: Pareto optimization for
  conflicting objectives
\item
  \textbf{Explainable Optimization}: Interpretable and transparent
  optimization processes
\end{itemize}

\section{Technical Ecosystem \& Tools}\label{technical-ecosystem-tools}

\subsection{Why Python for
Optimization?}\label{why-python-for-optimization}

Python has emerged as the \textbf{lingua franca} for computational
optimization due to:

\begin{itemize}
\tightlist
\item
  \textbf{Rich Ecosystem}: Comprehensive libraries for every
  optimization paradigm
\item
  \textbf{AI/ML Integration}: Seamless connection with machine learning
  frameworks
\item
  \textbf{Performance}: C/Fortran-backed numerical computing with Python
  simplicity
\item
  \textbf{Community}: Vibrant ecosystem with continuous algorithm
  development
\item
  \textbf{Industry Adoption}: Widely used in both academia and industry
\end{itemize}

\subsection{Core Toolchain}\label{core-toolchain}

Throughout this course, we'll work with industry-standard tools:

\textbf{Our Computational Optimization Stack:}

\begin{longtable}[]{@{}ll@{}}
\toprule\noalign{}
Category & Tools \\
\midrule\noalign{}
\endhead
\bottomrule\noalign{}
\endlastfoot
\textbf{Numerical Computing} & NumPy, SciPy \\
\textbf{Linear Programming} & PuLP, CVXPY \\
\textbf{Machine Learning} & Scikit-learn, TensorFlow, PyTorch \\
\textbf{Graph Algorithms} & NetworkX \\
\textbf{Visualization} & Matplotlib, Plotly, Seaborn \\
\textbf{Development} & Jupyter, VSCode, Git \\
\end{longtable}

\section{Getting Started: Your Learning
Journey}\label{getting-started-your-learning-journey}

\subsection{Prerequisites \&
Preparation}\label{prerequisites-preparation}

To succeed in this course, you should have:

\begin{itemize}
\tightlist
\item
  \textbf{Programming Fundamentals}: Basic Python proficiency
\item
  \textbf{Mathematical Background}: Linear algebra and calculus
  foundations\\
\item
  \textbf{Computational Mindset}: Willingness to experiment and debug
\item
  \textbf{Problem-Solving Attitude}: Persistence through challenging
  concepts
\item
  \textbf{Curiosity and Creativity}: Interest in exploring multiple
  solution approaches
\end{itemize}

\subsection{How to Maximize Your
Learning}\label{how-to-maximize-your-learning}

\begin{enumerate}
\def\labelenumi{\arabic{enumi}.}
\tightlist
\item
  \textbf{Engage Actively}: Don't just read---implement every concept in
  code
\item
  \textbf{Think Critically}: Question why certain methods work better
  for specific problems
\item
  \textbf{Experiment Freely}: Modify parameters, break code, and learn
  from failures
\item
  \textbf{Connect Concepts}: Relate theoretical principles to practical
  implementations
\item
  \textbf{Build Portfolio}: Document your work for future career
  opportunities
\item
  \textbf{Collaborate Effectively}: Learn from peers through code
  reviews and discussions
\item
  \textbf{Stay Updated}: Follow recent developments in optimization
  research
\end{enumerate}

\subsection{Course Structure and
Expectations}\label{course-structure-and-expectations}

This course is designed as a \textbf{blended learning experience}
combining:

\begin{itemize}
\tightlist
\item
  \textbf{Theoretical Foundations}: Mathematical principles and
  algorithm concepts
\item
  \textbf{Practical Implementation}: Hands-on coding exercises and
  projects
\item
  \textbf{Real-World Applications}: Industry-relevant case studies and
  problems
\item
  \textbf{Assessment and Feedback}: Continuous evaluation and
  improvement
\end{itemize}

\section{🌟 Welcome to the Journey}\label{welcome-to-the-journey}

You are beginning a journey into one of the most fundamental and
powerful domains of computer science and artificial intelligence. The
skills you develop here will serve as a foundation for advanced work in
machine learning, operations research, data science, and algorithmic
design.

As we progress through the modules, remember that each concept builds
toward a comprehensive understanding of how to make computers not just
compute, but \textbf{optimize}---transforming them from calculators into
intelligent decision-makers.

The journey through computational optimization is challenging but
immensely rewarding. You'll gain not just technical skills, but a new
way of thinking about problem-solving that will serve you throughout
your career in technology.

\begin{quote}
\textbf{``Optimization is the science of better. In a world of limited
resources and unlimited wants, optimization provides the mathematical
foundation for making the best possible decisions.''}
\end{quote}

\begin{center}\rule{0.5\linewidth}{0.5pt}\end{center}

\bookmarksetup{startatroot}

\chapter{Module 1: Basics of Optimization and Linear
Programming}\label{module-1-basics-of-optimization-and-linear-programming}

\section{Module Overview}\label{module-overview}

\subsection{Learning Objectives}\label{learning-objectives}

\begin{itemize}
\tightlist
\item
  Understand fundamental optimization concepts and problem
  classification
\item
  Formulate real-world problems as Linear Programming models
\item
  Solve LP problems using graphical and computational methods
\item
  Implement LP solutions in Python for practical applications
\item
  Apply optimization thinking to facility location problems
\end{itemize}

\subsection{Smart City Context: Warehouse Location
Challenge}\label{smart-city-context-warehouse-location-challenge}

In our Smart City Logistics project, we face a critical business
decision: \emph{where to locate new distribution warehouses} to minimize
transportation costs while serving all city facilities efficiently. This
module provides the mathematical foundation to solve this strategic
problem.

\section{Foundations of Optimization}\label{foundations-of-optimization}

\subsection{What is Optimization?}\label{what-is-optimization}

Optimization is the science of finding the \emph{best possible solution}
from all feasible alternatives under given constraints. In computational
terms, it involves:

\begin{itemize}
\tightlist
\item
  \textbf{Decision Variables}: Quantities we can control (e.g.,
  warehouse locations, shipment quantities)
\item
  \textbf{Objective Function}: What we want to maximize or minimize
  (e.g., total transportation cost)
\item
  \textbf{Constraints}: Limitations and requirements (e.g., budget,
  capacity, demand)
\end{itemize}

\subsection{Optimization Problem
Classification}\label{optimization-problem-classification}

\begin{longtable}[]{@{}
  >{\raggedright\arraybackslash}p{(\linewidth - 4\tabcolsep) * \real{0.2653}}
  >{\raggedright\arraybackslash}p{(\linewidth - 4\tabcolsep) * \real{0.3469}}
  >{\raggedright\arraybackslash}p{(\linewidth - 4\tabcolsep) * \real{0.3878}}@{}}
\toprule\noalign{}
\begin{minipage}[b]{\linewidth}\raggedright
Problem Type
\end{minipage} & \begin{minipage}[b]{\linewidth}\raggedright
Characteristics
\end{minipage} & \begin{minipage}[b]{\linewidth}\raggedright
Smart City Example
\end{minipage} \\
\midrule\noalign{}
\endhead
\bottomrule\noalign{}
\endlastfoot
\textbf{Linear Programming} & Linear objective and constraints &
Warehouse location with fixed costs \\
\textbf{Nonlinear Programming} & Nonlinear relationships & Fuel costs
that increase with distance \\
\textbf{Constrained Optimization} & With limitations & Budget
constraints on construction \\
\textbf{Unconstrained Optimization} & No limitations & Theoretical ideal
locations \\
\textbf{Discrete Optimization} & Integer decisions & Yes/no decisions
for locations \\
\textbf{Continuous Optimization} & Real-valued decisions & Precise
coordinates for facilities \\
\end{longtable}

\subsection{Real-World Applications in
CSE}\label{real-world-applications-in-cse}

\begin{itemize}
\tightlist
\item
  \textbf{Resource Allocation}: CPU time, memory allocation in operating
  systems
\item
  \textbf{Network Optimization}: Internet routing, data flow
  optimization
\item
  \textbf{Machine Learning}: Model training via loss function
  minimization
\item
  \textbf{Database Systems}: Query optimization and indexing
\item
  \textbf{Computer Graphics}: Rendering optimization and path tracing
\end{itemize}

\section{Basic Python for
Optimization}\label{basic-python-for-optimization}

\subsection{Essential Python Libraries
Setup}\label{essential-python-libraries-setup}

\begin{Shaded}
\begin{Highlighting}[]
\CommentTok{\# Core optimization stack installation}
\CommentTok{\# pip install numpy scipy matplotlib pulp pandas jupyter}
\end{Highlighting}
\end{Shaded}

\begin{tcolorbox}[enhanced jigsaw, breakable, coltitle=black, colframe=quarto-callout-tip-color-frame, arc=.35mm, bottomtitle=1mm, opacityback=0, toprule=.15mm, rightrule=.15mm, colback=white, bottomrule=.15mm, toptitle=1mm, titlerule=0mm, left=2mm, opacitybacktitle=0.6, colbacktitle=quarto-callout-tip-color!10!white, title=\textcolor{quarto-callout-tip-color}{\faLightbulb}\hspace{0.5em}{Key \texttt{Python} Libraries Required for Computational Part}, leftrule=.75mm]

\begin{itemize}
\item
  \texttt{NumPy}: Numerical computing and array operations
\item
  \texttt{SciPy}: Scientific computing and optimization algorithms
\item
  \texttt{Matplotlib}: Data visualization and result plotting
\item
  \texttt{PuLP}: Linear programming interface
\item
  \texttt{Pandas}: Data manipulation and analysis
\end{itemize}

\end{tcolorbox}

\subsection{Python Fundamentals for
Optimization}\label{python-fundamentals-for-optimization}

\begin{Shaded}
\begin{Highlighting}[]
\CommentTok{\# Essential operations we\textquotesingle{}ll use frequently}
\ImportTok{import}\NormalTok{ numpy }\ImportTok{as}\NormalTok{ np}
\ImportTok{import}\NormalTok{ matplotlib.pyplot }\ImportTok{as}\NormalTok{ plt}

\CommentTok{\# Array operations for constraint matrices}
\NormalTok{coefficients }\OperatorTok{=}\NormalTok{ np.array([[}\DecValTok{2}\NormalTok{, }\DecValTok{1}\NormalTok{], [}\DecValTok{1}\NormalTok{, }\DecValTok{3}\NormalTok{], [}\DecValTok{4}\NormalTok{, }\DecValTok{2}\NormalTok{]])}

\CommentTok{\# Function definitions for objectives}
\KeywordTok{def}\NormalTok{ transportation\_cost(warehouses, facilities):}
    \ControlFlowTok{return}\NormalTok{ np.}\BuiltInTok{sum}\NormalTok{(warehouses }\OperatorTok{*}\NormalTok{ facilities)}

\CommentTok{\# Data handling for problem parameters}
\NormalTok{demand\_data }\OperatorTok{=}\NormalTok{ \{}\StringTok{\textquotesingle{}Hospital\textquotesingle{}}\NormalTok{: }\DecValTok{50}\NormalTok{, }\StringTok{\textquotesingle{}School\textquotesingle{}}\NormalTok{: }\DecValTok{30}\NormalTok{, }\StringTok{\textquotesingle{}Mall\textquotesingle{}}\NormalTok{: }\DecValTok{80}\NormalTok{\}}
\end{Highlighting}
\end{Shaded}

\section{Linear Programming
Fundamentals}\label{linear-programming-fundamentals}

\subsection{Mathematical Definition of Linear
Programming}\label{mathematical-definition-of-linear-programming}

A Linear Programming (LP) problem can be formally defined as:

\textbf{Objective Function:} \[
\text{Optimize } Z = c_1x_1 + c_2x_2 + \cdots + c_nx_n
\]

\textbf{Subject to Constraints:}

\[
\begin{aligned}
a_{11}x_1 + a_{12}x_2 + \cdots + a_{1n}x_n & \leq b_1 \\
a_{21}x_1 + a_{22}x_2 + \cdots + a_{2n}x_n & \leq b_2 \\
\vdots & \\
a_{m1}x_1 + a_{m2}x_2 + \cdots + a_{mn}x_n & \leq b_m \\
\end{aligned}
\]

\textbf{Non-negativity Conditions:}

\[
x_1, x_2, \ldots, x_n \geq 0
\]

Where:

\begin{itemize}
\tightlist
\item
  \(Z\) is the objective function to be optimized (maximized or
  minimized)
\item
  \(x_1, x_2, \ldots, x_n\) are the decision variables
\item
  \(c_1, c_2, \ldots, c_n\) are the coefficients of the objective
  function
\item
  \(a_{ij}\) are the technological coefficients
\item
  \(b_1, b_2, \ldots, b_m\) are the right-hand side constants
\end{itemize}

\subsection{Standard Forms of Linear
Programming}\label{standard-forms-of-linear-programming}

\subsubsection{Canonical Form
(Maximization)}\label{canonical-form-maximization}

\[
\begin{aligned}
\text{Maximize } & Z = \mathbf{c}^T\mathbf{x} \\
\text{Subject to } & \mathbf{A}\mathbf{x} \leq \mathbf{b} \\
& \mathbf{x} \geq \mathbf{0}
\end{aligned}
\]

\subsubsection{Standard Form
(Maximization)}\label{standard-form-maximization}

\[
\begin{aligned}
\text{Maximize } & Z = \mathbf{c}^T\mathbf{x} \\
\text{Subject to } & \mathbf{A}\mathbf{x} = \mathbf{b} \\
& \mathbf{x} \geq \mathbf{0}
\end{aligned}
\]

\subsection{Key Concepts in Linear
Programming}\label{key-concepts-in-linear-programming}

\subsubsection{Feasible Region}\label{feasible-region}

The set of all points that satisfy all constraints simultaneously: \[
F = \{\mathbf{x} \in \mathbb{R}^n : \mathbf{A}\mathbf{x} \leq \mathbf{b}, \mathbf{x} \geq \mathbf{0}\}
\]

\subsubsection{Optimal Solution}\label{optimal-solution}

A point (\mathbf{x}\^{}*) in the feasible region that gives the best
value of the objective function.

\subsubsection{Corner Point (Extreme
Point)}\label{corner-point-extreme-point}

A point in the feasible region that cannot be expressed as a convex
combination of two other distinct points in the region.

\subsection{Smart City Case: Warehouse Location
Formulation}\label{smart-city-case-warehouse-location-formulation}

\subsubsection{Problem Statement}\label{problem-statement}

We need to choose locations for 2 warehouses from 4 potential sites to
serve 3 major facilities while minimizing total costs.

\subsubsection{Decision Variables}\label{decision-variables}

Let: - \(x_{ij}\): Amount shipped from warehouse \(i\) to facility \(j\)
- \(y_i\): Binary variable (1 if warehouse \(i\) is built, 0 otherwise)

Where: - \(i = 1, 2, 3, 4\) (warehouse sites) - \(j = 1, 2, 3\)
(facilities: Hospital, Mall, School)

\subsubsection{Objective Function}\label{objective-function}

Minimize total cost: \[
\text{Minimize } Z = \sum_{i=1}^{4} \sum_{j=1}^{3} c_{ij}x_{ij} + \sum_{i=1}^{4} f_i y_i
\] Where: - \(c_{ij}\): Transportation cost per unit from warehouse
\(i\) to facility \(j\) - \(f_i\): Fixed construction cost for warehouse
\(i\)

\subsubsection{Constraints}\label{constraints}

\textbf{Demand Constraints} (Each facility's demand must be met): \[
\sum_{i=1}^{4} x_{ij} = d_j \quad \text{for } j = 1,2,3
\] Where \(d_j\) is the demand of facility \(j\).

\textbf{Capacity Constraints} (Warehouse capacity cannot be exceeded):
\[
\sum_{j=1}^{3} x_{ij} \leq C_i y_i \quad \text{for } i = 1,2,3,4
\] Where \(C_i\) is the capacity of warehouse \(i\).

\textbf{Budget Constraint}: \[
\sum_{i=1}^{4} f_i y_i \leq B
\] Where \(B\) is the total budget available.

\textbf{Warehouse Selection Constraint} (Select exactly 2 warehouses):
\[
\sum_{i=1}^{4} y_i = 2
\]

\textbf{Non-negativity and Binary Requirements}: \[
x_{ij} \geq 0 \quad \text{for all } i,j
\] \[
y_i \in \{0, 1\} \quad \text{for all } i
\]

\subsection{Graphical Method for Two-Variable
Problems}\label{graphical-method-for-two-variable-problems}

\subsubsection{Step-by-Step Procedure}\label{step-by-step-procedure}

\begin{enumerate}
\def\labelenumi{\arabic{enumi}.}
\tightlist
\item
  \textbf{Plot Constraints}: Convert each inequality to equality and
  plot the line
\item
  \textbf{Identify Feasible Region}: Determine which side of each line
  satisfies the inequality
\item
  \textbf{Find Corner Points}: Identify intersection points of
  constraint boundaries
\item
  \textbf{Evaluate Objective Function}: Calculate objective value at
  each corner point
\item
  \textbf{Identify Optimal Solution}: Select the corner point with best
  objective value
\end{enumerate}

\subsubsection{Example: Simple Warehouse
Problem}\label{example-simple-warehouse-problem}

Consider a simplified version with 2 warehouses and 1 facility:

\textbf{Problem:} \[
\begin{aligned}
\text{Maximize } & Z = 40x_1 + 30x_2 \\
\text{Subject to } & 2x_1 + x_2 \leq 100 \\
& x_1 + 3x_2 \leq 90 \\
& x_1 \geq 0, x_2 \geq 0
\end{aligned}
\]

\textbf{Step 1: Plot Constraints} - Constraint 1: \(2x_1 + x_2 = 100\) →
Line through (0,100) and (50,0) - Constraint 2: \(x_1 + 3x_2 = 90\) →
Line through (0,30) and (90,0)

\textbf{Step 2: Identify Corner Points} - Intersection of axes: (0,0),
(0,30), (50,0) - Intersection of constraints: Solve: \[
  \begin{aligned}
  2x_1 + x_2 &= 100 \\
  x_1 + 3x_2 &= 90
  \end{aligned}
 \] Solution: \(x_1 = 42, x_2 = 16\)

\textbf{Step 3: Evaluate Objective Function}

\begin{itemize}
\tightlist
\item
  At (0,0): \(Z = 0\)
\item
  At (0,30): \(Z = 900\)
\item
  At (50,0): \(Z = 2000\)
\item
  At (42,16): \(Z = 40×42 + 30×16 = 2160\)
\end{itemize}

\textbf{Optimal Solution}: \(x_1 = 42, x_2 = 16\) with \(Z = 2160\)

\subsection{Simplex Method Theory}\label{simplex-method-theory}

\subsubsection{Fundamental Theorem of Linear
Programming}\label{fundamental-theorem-of-linear-programming}

If an LP problem has an optimal solution, then there exists at least one
corner point of the feasible region that is optimal.

\subsubsection{Simplex Algorithm Steps}\label{simplex-algorithm-steps}

\begin{enumerate}
\def\labelenumi{\arabic{enumi}.}
\item
  \textbf{Convert to Standard Form}

  \begin{itemize}
  \tightlist
  \item
    Convert inequalities to equations using slack variables
  \item
    Ensure all variables are non-negative
  \item
    Convert minimization to maximization if needed
  \end{itemize}
\item
  \textbf{Initial Basic Feasible Solution}

  \begin{itemize}
  \tightlist
  \item
    Identify basic variables (variables with coefficient 1 in one
    equation and 0 in others)
  \item
    Set non-basic variables to zero
  \end{itemize}
\item
  \textbf{Optimality Test}

  \begin{itemize}
  \tightlist
  \item
    Calculate reduced costs for non-basic variables
  \item
    If all reduced costs \(\leq 0\) (for maximization), current solution
    is optimal
  \end{itemize}
\item
  \textbf{Pivot Column Selection}

  \begin{itemize}
  \tightlist
  \item
    Choose non-basic variable with most positive reduced cost (for
    maximization)
  \end{itemize}
\item
  \textbf{Pivot Row Selection}

  \begin{itemize}
  \tightlist
  \item
    Use minimum ratio test:
    \(\min \left\{ \frac{b_i}{a_{ij}} : a_{ij} > 0 \right\}\)
  \end{itemize}
\item
  \textbf{Pivot Operation}

  \begin{itemize}
  \tightlist
  \item
    Make pivot element equal to 1
  \item
    Make other elements in pivot column equal to 0
  \end{itemize}
\item
  \textbf{Repeat} until optimality condition is satisfied
\end{enumerate}

\subsubsection{Simplex Tableau Format}\label{simplex-tableau-format}

The simplex method uses a tableau to organize calculations:

\begin{longtable}[]{@{}llllll@{}}
\toprule\noalign{}
Basic Var & \(x_1\) & \(x_2\) & \(s_1\) & \(s_2\) & RHS \\
\midrule\noalign{}
\endhead
\bottomrule\noalign{}
\endlastfoot
\(s_1\) & 2 & 1 & 1 & 0 & 100 \\
\(s_2\) & 1 & 3 & 0 & 1 & 90 \\
\(Z\) & -40 & -30 & 0 & 0 & 0 \\
\end{longtable}

Where \(s_1, s_2\) are slack variables.

\subsection{Types of Linear Programming
Solutions}\label{types-of-linear-programming-solutions}

\subsubsection{Unique Optimal Solution}\label{unique-optimal-solution}

When only one corner point gives the optimal objective value.

\subsubsection{Multiple Optimal
Solutions}\label{multiple-optimal-solutions}

When multiple corner points give the same optimal objective value.

\subsubsection{Unbounded Solution}\label{unbounded-solution}

When the objective function can be improved indefinitely.

\subsubsection{Infeasible Problem}\label{infeasible-problem}

When no point satisfies all constraints simultaneously.

\subsection{Duality in Linear
Programming}\label{duality-in-linear-programming}

Every LP problem (primal) has a corresponding dual problem:

\textbf{Primal Problem:} \[
\begin{aligned}
\text{Maximize } & Z = \mathbf{c}^T\mathbf{x} \\
\text{Subject to } & \mathbf{A}\mathbf{x} \leq \mathbf{b} \\
& \mathbf{x} \geq \mathbf{0}
\end{aligned}
\]

\textbf{Dual Problem:} \[
\begin{aligned}
\text{Minimize } & W = \mathbf{b}^T\mathbf{y} \\
\text{Subject to } & \mathbf{A}^T\mathbf{y} \geq \mathbf{c} \\
& \mathbf{y} \geq \mathbf{0}
\end{aligned}
\]

\subsubsection{Duality Theorems}\label{duality-theorems}

\begin{enumerate}
\def\labelenumi{\arabic{enumi}.}
\item
  \textbf{Weak Duality Theorem}: For any feasible solution
  \(\mathbf{x}\) of primal and \(\mathbf{y}\) of dual: \[
  \mathbf{c}^T\mathbf{x} \leq \mathbf{b}^T\mathbf{y}
    \]
\item
  \textbf{Strong Duality Theorem}: If either primal or dual has an
  optimal solution, then both have optimal solutions and: \[
  \mathbf{c}^T\mathbf{x}^* = \mathbf{b}^T\mathbf{y}^*
    \]
\end{enumerate}

\subsection{Sensitivity Analysis}\label{sensitivity-analysis}

Sensitivity analysis examines how changes in parameters affect the
optimal solution:

\subsubsection{Changes in Objective Function
Coefficients}\label{changes-in-objective-function-coefficients}

\begin{itemize}
\tightlist
\item
  Range of optimality for each coefficient
\item
  Effect on optimal solution when coefficients change
\end{itemize}

\subsubsection{Changes in Right-Hand Side
Constants}\label{changes-in-right-hand-side-constants}

\begin{itemize}
\tightlist
\item
  Shadow prices (dual variables)
\item
  Range of feasibility for each constraint
\end{itemize}

\subsubsection{Changes in Constraint
Coefficients}\label{changes-in-constraint-coefficients}

\begin{itemize}
\tightlist
\item
  Effect of adding new variables or constraints
\item
  Impact of changing technological coefficients
\end{itemize}

\subsection{Special Cases in Linear
Programming}\label{special-cases-in-linear-programming}

\subsubsection{Transportation Problems}\label{transportation-problems}

Special structure where sources supply destinations with minimum
transportation cost.

\subsubsection{Assignment Problems}\label{assignment-problems}

Special case of transportation problem where each source is assigned to
exactly one destination.

\subsubsection{Network Flow Problems}\label{network-flow-problems}

Optimization problems defined on networks with flow conservation
constraints.

\section{Linear Programming Problems --- Graphical
Method}\label{linear-programming-problems-graphical-method}

Below are \textbf{five LPP problems} designed to be solved using the
\textbf{graphical method}. For each problem:

\begin{enumerate}
\def\labelenumi{\arabic{enumi}.}
\tightlist
\item
  Draw each constraint as a straight line in the \(x_1\)--\(x_2\)
  plane.\\
\item
  Identify the feasible region (including \(x_1,x_2\ge0\)).\\
\item
  Find all corner (vertex) points of the feasible region.\\
\item
  Evaluate the objective \(Z\) at each corner to determine the
  optimum.\\
\item
  State the optimal solution and the optimal value \(Z^\star\).
\end{enumerate}

\begin{quote}
\textbf{Problem 1} --- Simple maximization
\end{quote}

\textbf{Maximize} \[
Z = 3x_1 + 2x_2
\]

\textbf{Subject to} \[
\begin{aligned}
x_1 + x_2 &\le 4,\\
x_1 &\le 3,\\
x_2 &\le 2,\\
x_1, x_2 &\ge 0.
\end{aligned}
\]

\begin{quote}
\textbf{Problem 2} --- Two binding inequalities
\end{quote}

\textbf{Maximize} \[
Z = 5x_1 + 4x_2
\]

\textbf{Subject to} \[
\begin{aligned}
2x_1 + x_2 &\le 8,\\
x_1 + 2x_2 &\le 8,\\
x_1, x_2 &\ge 0.
\end{aligned}
\]

\begin{quote}
\textbf{Problem 3} --- Minimization with intersection corner
\end{quote}

\textbf{Minimize} \[
Z = 4x_1 + 6x_2
\]

\textbf{Subject to} \[
\begin{aligned}
x_1 + x_2 &\ge 4,\\
2x_1 + x_2 &\ge 6,\\
x_1, x_2 &\ge 0.
\end{aligned}
\]

\begin{quote}
\textbf{Problem 4} --- Resource allocation (mix of \(\le\) and \(\ge\))
\end{quote}

A factory makes two products \(P_1\) and \(P_2\) with profits \$7 and
\$5 respectively.

\textbf{Maximize profit} \[
Z = 7x_1 + 5x_2
\] \textbf{Subject to resource limits} \[
\begin{aligned}
3x_1 + 2x_2 &\le 18 \quad (\text{machine-hours}),\\
x_1 + 2x_2 &\ge 4 \quad (\text{minimum production constraint}),\\
x_1, x_2 &\ge 0.
\end{aligned}
\]

\begin{quote}
\textbf{Problem 5} --- Problem with a redundant constraint
\end{quote}

\textbf{Maximize} \[
Z = 2x_1 + 3x_2
\]

\textbf{Subject to} \[
\begin{aligned}
x_1 + 4x_2 &\le 12,\\
2x_1 + 8x_2 &\le 24,\\
x_1 + x_2 &\le 5,\\
x_1, x_2 &\ge 0.
\end{aligned}
\]

\section{Practical Implementation with
Python}\label{practical-implementation-with-python}

\begin{quote}
\emph{Problem Statement}
\end{quote}

A Smart City logistics company needs to determine the optimal
distribution strategy between two warehouses (Warehouse A and Warehouse
B) to minimize total operational costs. The company must decide how many
units to ship from each warehouse while considering capacity constraints
and transportation costs.

\subsubsection{Problem Data:}\label{problem-data}

\begin{itemize}
\tightlist
\item
  \textbf{Warehouse A}: Transportation cost = \$40 per unit
\item
  \textbf{Warehouse B}: Transportation cost = \$30 per unit
\item
  \textbf{Constraint 1}: Minimum demand exeeds 60
\item
  \textbf{Constraint 2}: Warehouse A has a maximum capacity of 50
\item
  \textbf{Constraint 3}: Warehouse B has a maximum capacity of 40
\item
  \textbf{Constraint 4}: Total units from both warehouses cannot exceed
  200 (2 units from A + 3 unit from B combination)
\item
  Both warehouses have non-negative shipment quantities
\end{itemize}

\subsection{Mathematical Formulation}\label{mathematical-formulation}

\subsubsection{Decision Variables}\label{decision-variables-1}

Let: - \$ x\_1 \$: Number of units to ship from Warehouse A - \$ x\_2
\$: Number of units to ship from Warehouse B

\subsubsection{Objective Function}\label{objective-function-1}

Minimize the total transportation cost: \[
\text{Minimize } Z = 40x_1 + 30x_2
\]

\subsection{Complete Linear Programming
Model}\label{complete-linear-programming-model}

\[
\begin{aligned}
\text{Minimize } & Z = 40x_1 + 30x_2 \\
\text{Subject to } & x_1+x_2\geq 60\\
&x_1\leq 50\\
&x_2\leq 40\\
&2x_1 + 3x_2 \leq 200 \\
& x_1 \geq 0 \\
& x_2 \geq 0
\end{aligned}
\]

\begin{quote}
\emph{Using PuLP for LP Problems}
\end{quote}

\texttt{PuLP} provides a high-level interface for formulating and
solving optimization problems:

\begin{Shaded}
\begin{Highlighting}[]
\ImportTok{import}\NormalTok{ pulp}

\CommentTok{\# Create optimization problem}
\NormalTok{model }\OperatorTok{=}\NormalTok{ pulp.LpProblem(}\StringTok{"Warehouse\_Location"}\NormalTok{, pulp.LpMinimize)}

\CommentTok{\# Define decision variables}
\NormalTok{x1 }\OperatorTok{=}\NormalTok{ pulp.LpVariable(}\StringTok{\textquotesingle{}Warehouse\_A\textquotesingle{}}\NormalTok{, lowBound}\OperatorTok{=}\DecValTok{0}\NormalTok{, cat}\OperatorTok{=}\StringTok{\textquotesingle{}Continuous\textquotesingle{}}\NormalTok{)}
\NormalTok{x2 }\OperatorTok{=}\NormalTok{ pulp.LpVariable(}\StringTok{\textquotesingle{}Warehouse\_B\textquotesingle{}}\NormalTok{, lowBound}\OperatorTok{=}\DecValTok{0}\NormalTok{, cat}\OperatorTok{=}\StringTok{\textquotesingle{}Continuous\textquotesingle{}}\NormalTok{)}

\CommentTok{\# Define objective function}
\NormalTok{model }\OperatorTok{+=} \DecValTok{40}\OperatorTok{*}\NormalTok{x1 }\OperatorTok{+} \DecValTok{30}\OperatorTok{*}\NormalTok{x2, }\StringTok{"Total\_Cost"}

\CommentTok{\# Add realistic constraints}
\NormalTok{model }\OperatorTok{+=}\NormalTok{ x1 }\OperatorTok{+}\NormalTok{ x2 }\OperatorTok{\textgreater{}=} \DecValTok{60}\NormalTok{, }\StringTok{"Minimum\_Demand"}
\NormalTok{model }\OperatorTok{+=}\NormalTok{ x1 }\OperatorTok{\textless{}=} \DecValTok{50}\NormalTok{, }\StringTok{"Warehouse\_A\_Capacity"}  \CommentTok{\# Warehouse A max capacity}
\NormalTok{model }\OperatorTok{+=}\NormalTok{ x2 }\OperatorTok{\textless{}=} \DecValTok{40}\NormalTok{, }\StringTok{"Warehouse\_B\_Capacity"}  \CommentTok{\# Warehouse B max capacity}
\NormalTok{model }\OperatorTok{+=} \DecValTok{2}\OperatorTok{*}\NormalTok{x1 }\OperatorTok{+} \DecValTok{3}\OperatorTok{*}\NormalTok{x2 }\OperatorTok{\textless{}=} \DecValTok{200}\NormalTok{, }\StringTok{"Budget\_Constraint"}  \CommentTok{\# Combined resource constraint}

\CommentTok{\# Solve the problem}
\NormalTok{model.solve()}

\CommentTok{\# Print results}
\BuiltInTok{print}\NormalTok{(}\SpecialStringTok{f"Status: }\SpecialCharTok{\{}\NormalTok{pulp}\SpecialCharTok{.}\NormalTok{LpStatus[model.status]}\SpecialCharTok{\}}\SpecialStringTok{"}\NormalTok{)}
\BuiltInTok{print}\NormalTok{(}\SpecialStringTok{f"Optimal Cost: $}\SpecialCharTok{\{}\NormalTok{pulp}\SpecialCharTok{.}\NormalTok{value(model.objective)}\SpecialCharTok{:.2f\}}\SpecialStringTok{"}\NormalTok{)}
\BuiltInTok{print}\NormalTok{(}\SpecialStringTok{f"Warehouse A: }\SpecialCharTok{\{}\NormalTok{x1}\SpecialCharTok{.}\NormalTok{varValue}\SpecialCharTok{\}}\SpecialStringTok{ units"}\NormalTok{)}
\BuiltInTok{print}\NormalTok{(}\SpecialStringTok{f"Warehouse B: }\SpecialCharTok{\{}\NormalTok{x2}\SpecialCharTok{.}\NormalTok{varValue}\SpecialCharTok{\}}\SpecialStringTok{ units"}\NormalTok{)}
\end{Highlighting}
\end{Shaded}

\begin{verbatim}
Status: Optimal
Optimal Cost: $2000.00
Warehouse A: 20.0 units
Warehouse B: 40.0 units
\end{verbatim}

\section{Linear Programming Problems
Collection}\label{linear-programming-problems-collection}

\subsection{Problem 1}\label{problem-1}

A manufacturing company produces two products (A and B) using three
machines (M1, M2, M3). The profit per unit is \$50 for product A and
\$40 for product B. Machine time requirements and availability are given
below. Determine the optimal production quantities to maximize profit.

\begin{longtable}[]{@{}
  >{\raggedright\arraybackslash}p{(\linewidth - 6\tabcolsep) * \real{0.1343}}
  >{\raggedright\arraybackslash}p{(\linewidth - 6\tabcolsep) * \real{0.3134}}
  >{\raggedright\arraybackslash}p{(\linewidth - 6\tabcolsep) * \real{0.3134}}
  >{\raggedright\arraybackslash}p{(\linewidth - 6\tabcolsep) * \real{0.2388}}@{}}
\toprule\noalign{}
\begin{minipage}[b]{\linewidth}\raggedright
Machine
\end{minipage} & \begin{minipage}[b]{\linewidth}\raggedright
Product A (hrs/unit)
\end{minipage} & \begin{minipage}[b]{\linewidth}\raggedright
Product B (hrs/unit)
\end{minipage} & \begin{minipage}[b]{\linewidth}\raggedright
Available Hours
\end{minipage} \\
\midrule\noalign{}
\endhead
\bottomrule\noalign{}
\endlastfoot
M1 & 2 & 1 & 100 \\
M2 & 1 & 2 & 80 \\
M3 & 1 & 1 & 60 \\
\end{longtable}

\textbf{Mathematical Formulation}

\textbf{Decision Variables:}

\begin{itemize}
\tightlist
\item
  \(x_1\): Units of Product A to produce
\item
  \(x_2\): Units of Product B to produce
\end{itemize}

\textbf{Objective Function:} \[
\text{Maximize } Z = 50x_1 + 40x_2
\]

\textbf{Constraints:} \[
\begin{aligned}
2x_1 + x_2 &\leq 100 \quad \text{(Machine M1)} \\
x_1 + 2x_2 &\leq 80 \quad \text{(Machine M2)} \\
x_1 + x_2 &\leq 60 \quad \text{(Machine M3)} \\
x_1, x_2 &\geq 0
\end{aligned}
\]

\begin{quote}
\texttt{Python} Solution
\end{quote}

\begin{Shaded}
\begin{Highlighting}[]
\ImportTok{import}\NormalTok{ pulp}

\CommentTok{\# Create the optimization problem}
\NormalTok{model }\OperatorTok{=}\NormalTok{ pulp.LpProblem(}\StringTok{"Product\_Mix\_Optimization"}\NormalTok{, pulp.LpMaximize)}

\CommentTok{\# Decision variables}
\NormalTok{x1 }\OperatorTok{=}\NormalTok{ pulp.LpVariable(}\StringTok{\textquotesingle{}Product\_A\textquotesingle{}}\NormalTok{, lowBound}\OperatorTok{=}\DecValTok{0}\NormalTok{, cat}\OperatorTok{=}\StringTok{\textquotesingle{}Continuous\textquotesingle{}}\NormalTok{)}
\NormalTok{x2 }\OperatorTok{=}\NormalTok{ pulp.LpVariable(}\StringTok{\textquotesingle{}Product\_B\textquotesingle{}}\NormalTok{, lowBound}\OperatorTok{=}\DecValTok{0}\NormalTok{, cat}\OperatorTok{=}\StringTok{\textquotesingle{}Continuous\textquotesingle{}}\NormalTok{)}

\CommentTok{\# Objective function}
\NormalTok{model }\OperatorTok{+=} \DecValTok{50}\OperatorTok{*}\NormalTok{x1 }\OperatorTok{+} \DecValTok{40}\OperatorTok{*}\NormalTok{x2, }\StringTok{"Total\_Profit"}

\CommentTok{\# Constraints}
\NormalTok{model }\OperatorTok{+=} \DecValTok{2}\OperatorTok{*}\NormalTok{x1 }\OperatorTok{+}\NormalTok{ x2 }\OperatorTok{\textless{}=} \DecValTok{100}\NormalTok{, }\StringTok{"Machine\_M1"}
\NormalTok{model }\OperatorTok{+=}\NormalTok{ x1 }\OperatorTok{+} \DecValTok{2}\OperatorTok{*}\NormalTok{x2 }\OperatorTok{\textless{}=} \DecValTok{80}\NormalTok{, }\StringTok{"Machine\_M2"}
\NormalTok{model }\OperatorTok{+=}\NormalTok{ x1 }\OperatorTok{+}\NormalTok{ x2 }\OperatorTok{\textless{}=} \DecValTok{60}\NormalTok{, }\StringTok{"Machine\_M3"}

\CommentTok{\# Solve}
\NormalTok{model.solve()}

\CommentTok{\# Results}
\BuiltInTok{print}\NormalTok{(}\StringTok{"=== PRODUCT MIX OPTIMIZATION ==="}\NormalTok{)}
\BuiltInTok{print}\NormalTok{(}\SpecialStringTok{f"Status: }\SpecialCharTok{\{}\NormalTok{pulp}\SpecialCharTok{.}\NormalTok{LpStatus[model.status]}\SpecialCharTok{\}}\SpecialStringTok{"}\NormalTok{)}
\BuiltInTok{print}\NormalTok{(}\SpecialStringTok{f"Optimal Profit: $}\SpecialCharTok{\{}\NormalTok{pulp}\SpecialCharTok{.}\NormalTok{value(model.objective)}\SpecialCharTok{:.2f\}}\SpecialStringTok{"}\NormalTok{)}
\BuiltInTok{print}\NormalTok{(}\SpecialStringTok{f"Product A: }\SpecialCharTok{\{}\NormalTok{x1}\SpecialCharTok{.}\NormalTok{varValue}\SpecialCharTok{\}}\SpecialStringTok{ units"}\NormalTok{)}
\BuiltInTok{print}\NormalTok{(}\SpecialStringTok{f"Product B: }\SpecialCharTok{\{}\NormalTok{x2}\SpecialCharTok{.}\NormalTok{varValue}\SpecialCharTok{\}}\SpecialStringTok{ units"}\NormalTok{)}
\BuiltInTok{print}\NormalTok{(}\SpecialStringTok{f"Machine M1 usage: }\SpecialCharTok{\{}\DecValTok{2}\OperatorTok{*}\NormalTok{x1}\SpecialCharTok{.}\NormalTok{varValue }\OperatorTok{+}\NormalTok{ x2}\SpecialCharTok{.}\NormalTok{varValue}\SpecialCharTok{\}}\SpecialStringTok{/100 hours"}\NormalTok{)}
\BuiltInTok{print}\NormalTok{(}\SpecialStringTok{f"Machine M2 usage: }\SpecialCharTok{\{}\NormalTok{x1}\SpecialCharTok{.}\NormalTok{varValue }\OperatorTok{+} \DecValTok{2}\OperatorTok{*}\NormalTok{x2}\SpecialCharTok{.}\NormalTok{varValue}\SpecialCharTok{\}}\SpecialStringTok{/80 hours"}\NormalTok{)}
\BuiltInTok{print}\NormalTok{(}\SpecialStringTok{f"Machine M3 usage: }\SpecialCharTok{\{}\NormalTok{x1}\SpecialCharTok{.}\NormalTok{varValue }\OperatorTok{+}\NormalTok{ x2}\SpecialCharTok{.}\NormalTok{varValue}\SpecialCharTok{\}}\SpecialStringTok{/60 hours"}\NormalTok{)}
\end{Highlighting}
\end{Shaded}

\begin{verbatim}
=== PRODUCT MIX OPTIMIZATION ===
Status: Optimal
Optimal Profit: $2800.00
Product A: 40.0 units
Product B: 20.0 units
Machine M1 usage: 100.0/100 hours
Machine M2 usage: 80.0/80 hours
Machine M3 usage: 60.0/60 hours
\end{verbatim}

\subsection{Problem 2: Diet Problem}\label{problem-2-diet-problem}

A nutritionist needs to design a minimum-cost diet that meets daily
nutritional requirements. Two foods are available with different
nutrient contents and costs. Determine the optimal food quantities.

\begin{longtable}[]{@{}
  >{\raggedright\arraybackslash}p{(\linewidth - 6\tabcolsep) * \real{0.1389}}
  >{\raggedright\arraybackslash}p{(\linewidth - 6\tabcolsep) * \real{0.2500}}
  >{\raggedright\arraybackslash}p{(\linewidth - 6\tabcolsep) * \real{0.2500}}
  >{\raggedright\arraybackslash}p{(\linewidth - 6\tabcolsep) * \real{0.3611}}@{}}
\toprule\noalign{}
\begin{minipage}[b]{\linewidth}\raggedright
Nutrient
\end{minipage} & \begin{minipage}[b]{\linewidth}\raggedright
Food 1 (units/kg)
\end{minipage} & \begin{minipage}[b]{\linewidth}\raggedright
Food 2 (units/kg)
\end{minipage} & \begin{minipage}[b]{\linewidth}\raggedright
Minimum Daily Requirement
\end{minipage} \\
\midrule\noalign{}
\endhead
\bottomrule\noalign{}
\endlastfoot
Protein & 2 & 1 & 8 units \\
Carbs & 1 & 2 & 10 units \\
Fat & 1 & 1 & 6 units \\
Cost (\$) & 3 & 2 & - \\
\end{longtable}

\textbf{Mathematical Formulation}

\begin{quote}
Decision Variables:
\end{quote}

\begin{itemize}
\tightlist
\item
  \(x_1\): kg of Food 1 to include in daily diet
\item
  \(x_2\): kg of Food 2 to include in daily diet
\end{itemize}

\begin{quote}
\textbf{Objective Function:}
\end{quote}

Minimize the total daily cost: \[
\text{Minimize } Z = 3x_1 + 2x_2
\]

\begin{quote}
\textbf{Constraints:}
\end{quote}

\emph{Protein Requirement:} \[
2x_1 + x_2 \geq 8
\]

\emph{Carbohydrates Requirement:} \[
x_1 + 2x_2 \geq 10
\]

\emph{Fat Requirement:} \[
x_1 + x_2 \geq 6
\]

\emph{Non-negativity Constraints:} \[
x_1 \geq 0, \quad x_2 \geq 0
\]

\textbf{Complete Linear Programming Model}

\[
\begin{aligned}
\text{Minimize } & Z = 3x_1 + 2x_2 \\
\text{Subject to } & 2x_1 + x_2 \geq 8 \\
& x_1 + 2x_2 \geq 10 \\
& x_1 + x_2 \geq 6 \\
& x_1 \geq 0 \\
& x_2 \geq 0
\end{aligned}
\]

\begin{quote}
\texttt{Python} Implementation
\end{quote}

\begin{Shaded}
\begin{Highlighting}[]
\ImportTok{import}\NormalTok{ pulp}

\CommentTok{\# Create the optimization problem}
\NormalTok{model }\OperatorTok{=}\NormalTok{ pulp.LpProblem(}\StringTok{"Diet\_Problem"}\NormalTok{, pulp.LpMinimize)}

\CommentTok{\# Define decision variables}
\NormalTok{x1 }\OperatorTok{=}\NormalTok{ pulp.LpVariable(}\StringTok{\textquotesingle{}Food\_1\textquotesingle{}}\NormalTok{, lowBound}\OperatorTok{=}\DecValTok{0}\NormalTok{, cat}\OperatorTok{=}\StringTok{\textquotesingle{}Continuous\textquotesingle{}}\NormalTok{)}
\NormalTok{x2 }\OperatorTok{=}\NormalTok{ pulp.LpVariable(}\StringTok{\textquotesingle{}Food\_2\textquotesingle{}}\NormalTok{, lowBound}\OperatorTok{=}\DecValTok{0}\NormalTok{, cat}\OperatorTok{=}\StringTok{\textquotesingle{}Continuous\textquotesingle{}}\NormalTok{)}

\CommentTok{\# Define objective function (minimize cost)}
\NormalTok{model }\OperatorTok{+=} \DecValTok{3}\OperatorTok{*}\NormalTok{x1 }\OperatorTok{+} \DecValTok{2}\OperatorTok{*}\NormalTok{x2, }\StringTok{"Total\_Cost"}

\CommentTok{\# Add nutritional constraints}
\NormalTok{model }\OperatorTok{+=} \DecValTok{2}\OperatorTok{*}\NormalTok{x1 }\OperatorTok{+}\NormalTok{ x2 }\OperatorTok{\textgreater{}=} \DecValTok{8}\NormalTok{, }\StringTok{"Protein\_Requirement"}
\NormalTok{model }\OperatorTok{+=}\NormalTok{ x1 }\OperatorTok{+} \DecValTok{2}\OperatorTok{*}\NormalTok{x2 }\OperatorTok{\textgreater{}=} \DecValTok{10}\NormalTok{, }\StringTok{"Carbohydrates\_Requirement"}
\NormalTok{model }\OperatorTok{+=}\NormalTok{ x1 }\OperatorTok{+}\NormalTok{ x2 }\OperatorTok{\textgreater{}=} \DecValTok{6}\NormalTok{, }\StringTok{"Fat\_Requirement"}

\CommentTok{\# Solve the problem}
\NormalTok{model.solve()}

\CommentTok{\# Display results}
\BuiltInTok{print}\NormalTok{(}\StringTok{"=== DIET PROBLEM OPTIMIZATION ==="}\NormalTok{)}
\BuiltInTok{print}\NormalTok{(}\SpecialStringTok{f"Solution Status: }\SpecialCharTok{\{}\NormalTok{pulp}\SpecialCharTok{.}\NormalTok{LpStatus[model.status]}\SpecialCharTok{\}}\SpecialStringTok{"}\NormalTok{)}
\BuiltInTok{print}\NormalTok{(}\SpecialStringTok{f"Minimum Daily Cost: $}\SpecialCharTok{\{}\NormalTok{pulp}\SpecialCharTok{.}\NormalTok{value(model.objective)}\SpecialCharTok{:.2f\}}\SpecialStringTok{"}\NormalTok{)}
\BuiltInTok{print}\NormalTok{(}\SpecialStringTok{f"Optimal Food Quantities:"}\NormalTok{)}
\BuiltInTok{print}\NormalTok{(}\SpecialStringTok{f"  Food 1: }\SpecialCharTok{\{}\NormalTok{x1}\SpecialCharTok{.}\NormalTok{varValue}\SpecialCharTok{:.2f\}}\SpecialStringTok{ kg"}\NormalTok{)}
\BuiltInTok{print}\NormalTok{(}\SpecialStringTok{f"  Food 2: }\SpecialCharTok{\{}\NormalTok{x2}\SpecialCharTok{.}\NormalTok{varValue}\SpecialCharTok{:.2f\}}\SpecialStringTok{ kg"}\NormalTok{)}

\CommentTok{\# Verify nutritional intake}
\BuiltInTok{print}\NormalTok{(}\SpecialStringTok{f"}\CharTok{\textbackslash{}n}\SpecialStringTok{Nutritional Analysis:"}\NormalTok{)}
\BuiltInTok{print}\NormalTok{(}\SpecialStringTok{f"Protein Intake: }\SpecialCharTok{\{}\DecValTok{2}\OperatorTok{*}\NormalTok{x1}\SpecialCharTok{.}\NormalTok{varValue }\OperatorTok{+}\NormalTok{ x2}\SpecialCharTok{.}\NormalTok{varValue}\SpecialCharTok{:.1f\}}\SpecialStringTok{ units (Minimum: 8 units)"}\NormalTok{)}
\BuiltInTok{print}\NormalTok{(}\SpecialStringTok{f"Carbohydrates Intake: }\SpecialCharTok{\{}\NormalTok{x1}\SpecialCharTok{.}\NormalTok{varValue }\OperatorTok{+} \DecValTok{2}\OperatorTok{*}\NormalTok{x2}\SpecialCharTok{.}\NormalTok{varValue}\SpecialCharTok{:.1f\}}\SpecialStringTok{ units (Minimum: 10 units)"}\NormalTok{)}
\BuiltInTok{print}\NormalTok{(}\SpecialStringTok{f"Fat Intake: }\SpecialCharTok{\{}\NormalTok{x1}\SpecialCharTok{.}\NormalTok{varValue }\OperatorTok{+}\NormalTok{ x2}\SpecialCharTok{.}\NormalTok{varValue}\SpecialCharTok{:.1f\}}\SpecialStringTok{ units (Minimum: 6 units)"}\NormalTok{)}

\CommentTok{\# Cost breakdown}
\BuiltInTok{print}\NormalTok{(}\SpecialStringTok{f"}\CharTok{\textbackslash{}n}\SpecialStringTok{Cost Breakdown:"}\NormalTok{)}
\BuiltInTok{print}\NormalTok{(}\SpecialStringTok{f"Food 1 Cost: $}\SpecialCharTok{\{}\DecValTok{3}\OperatorTok{*}\NormalTok{x1}\SpecialCharTok{.}\NormalTok{varValue}\SpecialCharTok{:.2f\}}\SpecialStringTok{"}\NormalTok{)}
\BuiltInTok{print}\NormalTok{(}\SpecialStringTok{f"Food 2 Cost: $}\SpecialCharTok{\{}\DecValTok{2}\OperatorTok{*}\NormalTok{x2}\SpecialCharTok{.}\NormalTok{varValue}\SpecialCharTok{:.2f\}}\SpecialStringTok{"}\NormalTok{)}
\BuiltInTok{print}\NormalTok{(}\SpecialStringTok{f"Total Cost: $}\SpecialCharTok{\{}\NormalTok{pulp}\SpecialCharTok{.}\NormalTok{value(model.objective)}\SpecialCharTok{:.2f\}}\SpecialStringTok{"}\NormalTok{)}
\end{Highlighting}
\end{Shaded}

\begin{verbatim}
=== DIET PROBLEM OPTIMIZATION ===
Solution Status: Optimal
Minimum Daily Cost: $14.00
Optimal Food Quantities:
  Food 1: 2.00 kg
  Food 2: 4.00 kg

Nutritional Analysis:
Protein Intake: 8.0 units (Minimum: 8 units)
Carbohydrates Intake: 10.0 units (Minimum: 10 units)
Fat Intake: 6.0 units (Minimum: 6 units)

Cost Breakdown:
Food 1 Cost: $6.00
Food 2 Cost: $8.00
Total Cost: $14.00
\end{verbatim}

\section{Micro-Project 1: Campus City Emergency Supply
Distribution}\label{micro-project-1-campus-city-emergency-supply-distribution}

As an optimization analyst at \textbf{Campus City Logistics}, you've
been tasked with designing the optimal supply distribution network for
essential resources across campus facilities. The current ad-hoc system
is inefficient and costly.

\subsection{Problem Statement}\label{problem-statement-1}

Determine the optimal warehouse locations and distribution plan that
minimizes total annual costs while meeting all facility demands,
respecting warehouse capacity constraints, and operating within the
allocated budget.

\subsection{Facilities Data (From
facilities.csv)}\label{facilities-data-from-facilities.csv}

Based on our dataset, we have 15 facilities with the following key
attributes:

\textbf{Critical Facilities Selected for Micro-Project 1:}

\begin{longtable}[]{@{}llll@{}}
\toprule\noalign{}
Facility ID & Facility Name & Type & Daily Demand \\
\midrule\noalign{}
\endhead
\bottomrule\noalign{}
\endlastfoot
MED\_CENTER & Campus Medical Center & Hospital & 80 units \\
ENG\_BUILDING & Engineering Building & Academic & 30 units \\
SCIENCE\_HALL & Science Hall & Academic & 35 units \\
DORM\_A & North Dormitory & Residential & 55 units \\
DORM\_B & South Dormitory & Residential & 45 units \\
LIBRARY & Main Library & Academic & 25 units \\
\end{longtable}

\textbf{Total Daily Demand: 270 units}

\subsection{Warehouse Data (From
warehouses.csv)}\label{warehouse-data-from-warehouses.csv}

\begin{longtable}[]{@{}
  >{\raggedright\arraybackslash}p{(\linewidth - 8\tabcolsep) * \real{0.1585}}
  >{\raggedright\arraybackslash}p{(\linewidth - 8\tabcolsep) * \real{0.1829}}
  >{\raggedright\arraybackslash}p{(\linewidth - 8\tabcolsep) * \real{0.1829}}
  >{\raggedright\arraybackslash}p{(\linewidth - 8\tabcolsep) * \real{0.2195}}
  >{\raggedright\arraybackslash}p{(\linewidth - 8\tabcolsep) * \real{0.2561}}@{}}
\toprule\noalign{}
\begin{minipage}[b]{\linewidth}\raggedright
Warehouse ID
\end{minipage} & \begin{minipage}[b]{\linewidth}\raggedright
Warehouse Name
\end{minipage} & \begin{minipage}[b]{\linewidth}\raggedright
Daily Capacity
\end{minipage} & \begin{minipage}[b]{\linewidth}\raggedright
Construction Cost
\end{minipage} & \begin{minipage}[b]{\linewidth}\raggedright
Operational Cost/Day
\end{minipage} \\
\midrule\noalign{}
\endhead
\bottomrule\noalign{}
\endlastfoot
WH\_NORTH & North Campus Warehouse & 400 units & \$300,000 & \$800 \\
WH\_SOUTH & South Campus Warehouse & 350 units & \$280,000 & \$700 \\
WH\_EAST & East Gate Warehouse & 450 units & \$320,000 & \$900 \\
\end{longtable}

\textbf{Total Available Capacity: 1,200 units}

\subsection{Transportation Costs (From
transportation\_costs.csv)}\label{transportation-costs-from-transportation_costs.csv}

\begin{itemize}
\tightlist
\item
  \textbf{Data Source}: Pre-calculated matrix with actual distances
\item
  \textbf{Cost Range}: \$3.68 - \$5.03 per unit between selected
  locations
\item
  \textbf{Calculation}: Based on real geographic coordinates using
  Haversine formula
\end{itemize}

\subsection{Financial Constraints}\label{financial-constraints}

\begin{itemize}
\tightlist
\item
  \textbf{Annual Budget}: \$1,500,000
\item
  \textbf{Operational Period}: 365 days (annual calculation)
\item
  \textbf{Construction Cost}: Amortized over 10 years
\item
  \textbf{All costs must be annualized}
\end{itemize}

\subsection{Physical \& Business
Constraints}\label{physical-business-constraints}

\begin{enumerate}
\def\labelenumi{\arabic{enumi}.}
\tightlist
\item
  \textbf{Warehouse Selection}: Select exactly 2 warehouses for
  redundancy
\item
  \textbf{Demand Satisfaction}: Each facility must receive exactly its
  daily demand × 365
\item
  \textbf{Capacity Limits}: Shipments from each warehouse ≤ capacity ×
  365
\item
  \textbf{Budget Limit}: Total annual cost ≤ \$1,500,000
\item
  \textbf{Non-negativity}: All shipment quantities ≥ 0
\end{enumerate}

\subsection{Learning Objectives}\label{learning-objectives-1}

\textbf{Technical Skills}

\begin{itemize}
\tightlist
\item
  Formulate Mixed-Integer Linear Programming (MILP) problem
\item
  Implement optimization model using PuLP
\item
  Handle real geographic and cost data
\item
  Validate constraint satisfaction
\end{itemize}

\textbf{Analytical Skills}

\begin{itemize}
\tightlist
\item
  Interpret optimization results in business context
\item
  Analyze cost breakdown and efficiency metrics
\item
  Provide actionable recommendations
\end{itemize}

\textbf{Project Deliverables}

\begin{enumerate}
\def\labelenumi{\arabic{enumi}.}
\tightlist
\item
  Deliverable 1: Mathematical Formulation (3 Marks)
\item
  Deliverable 2: Report in .pdf format (7 Marks)
\end{enumerate}




\end{document}
